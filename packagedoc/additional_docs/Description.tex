% --------------------------------------------------------------------------------------------------------------
%
% Copyright 2020-2024 Robert Bosch GmbH

% Licensed under the Apache License, Version 2.0 (the "License");
% you may not use this file except in compliance with the License.
% You may obtain a copy of the License at

% http://www.apache.org/licenses/LICENSE-2.0

% Unless required by applicable law or agreed to in writing, software
% distributed under the License is distributed on an "AS IS" BASIS,
% WITHOUT WARRANTIES OR CONDITIONS OF ANY KIND, either express or implied.
% See the License for the specific language governing permissions and
% limitations under the License.
%
% --------------------------------------------------------------------------------------------------------------

\section{Test setup}

How many tests are passed? How many tests are failed? On which test benches do these tests run? Did any resets occur during test execution?
How about the temporary unavailability of required test external components?

To monitor all these informations, a test system setup is necessary consisting of at least the following components:

\begin{enumerate}
   \item A test framework that executes the test and provides the data to be monitored (here: the \textbf{Robot Framework}).
         This includes the possibility that more than one test framework runs in parallel.

   \item A component that collects and stores data from all executed test frameworks (here: the monitoring system \textbf{Prometheus}).

   \item A component that visualizes all data that have been collected (here: \textbf{Grafana}).
\end{enumerate}

This is not the only way to establish such a test system, but in this document we concentrate on \textbf{Robot Framework}, \textbf{Prometheus} and \textbf{Grafana}.


To be able to collect values, \textbf{Prometheus} requires an http based counterpart. And the \textbf{Robot Framework} must be enabled to support this counterpart.
In pure Python this is realized by a \textbf{Prometheus Python client library}. On \textbf{Robot Framework} level this is done by this component \pkg\ that is a mapping
between the interface of the \textbf{Prometheus Python client library} and \textbf{Robot Framework} keywords.

Or in other words: \pkg\ uses the \textbf{Prometheus Python client library} to provide values to \textbf{Prometheus} to enable the data visualization in \textbf{Grafana}.

The \textbf{Prometheus Python client library} is part of the installation dependencies of \pkg. \textbf{Prometheus} and \textbf{Grafana} are components that
have to be downloaded, installed and configured separately.

% --------------------------------------------------------------------------------------------------------------

\section{Installations}

\textbf{1. Prometheus}

To install \textbf{Prometheus} please visit the \href{https://prometheus.io/}{homepage}. This homepage also contains a \textit{getting started} section containing useful hints
about how to configure the application.

Further informations can also be found \href{https://www.fullstackpython.com/prometheus.html}{here}.

\textbf{2. Grafana}

The advantage of using \textbf{Grafana} for data visualization is to have a \textit{ready to use} interface to \textbf{Prometheus} available. Other possible solutions are not in focus here.

To install \textbf{Grafana} please visit the \href{https://grafana.com/}{homepage}.

How to create a Prometheus data source in Grafana, is described \href{https://prometheus.io/docs/visualization/grafana/}{here}.

\textbf{3. Prometheus Python client library}

This library belongs to the installation dependencies of \pkg. A separate installation is not required. In case you want to learn more about this client library please visit the following web pages:
\href{https://pypi.org/project/prometheus-client/}{[1]}, \href{https://prometheus.github.io/client_python/}{[2]}, \href{https://prometheus.io/docs/prometheus/latest/getting_started/}{[3]}.

% --------------------------------------------------------------------------------------------------------------

\newpage

\section{Configuration}

\textbf{Prometheus} can be configured with a configuration file in YAML format. This includes the used port numbers. The example files in the \repo\ repository use
the port numbers 8000, 8001 and 8002. Therefore the configuration file of \textbf{Prometheus} requires the following entry:

\begin{pythoncode}
- targets: ["localhost:8000","localhost:8001","localhost:8002"]
\end{pythoncode}

% --------------------------------------------------------------------------------------------------------------

\section{Library import}

After installation (see \href{https://github.com/test-fullautomation/robotframework-prometheus/blob/develop/README.rst}{README}), the interface library
can be found within the \plog{site-packages} that is the usual place for installed Python modules. It is recommended to introduce an environment variable to store the
\plog{site-packages} path, e.g. \pcode{ROBOTPYTHONSITEPACKAGESPATH}.

Now within robot files the interface library can be imported in the following way:

\begin{robotcode}
*** Settings ***
Library    %{ROBOTPYTHONSITEPACKAGESPATH}/PrometheusInterface/prometheus_interface.py
\end{robotcode}

If nothing else is specified, the default port 8000 is used. It can be required to run several \textbf{Robot Framework} instances in parallel. In this case
every instance needs it's own port number. Within the \textbf{Prometheus} YAML file we already have three port numbers defined (like described above).

Let's assume now we want to execute testsuite \textit{A} parallel to testsuite \textit{B}. Then we can assign port number 8001 to testsuite \textit{A}:

\begin{robotcode}
*** Settings ***
Library    %{ROBOTPYTHONSITEPACKAGESPATH}/PrometheusInterface/prometheus_interface.py    port_number=${8001}
\end{robotcode}

and we assign port number 8002 to testsuite \textit{B}:

\begin{robotcode}
*** Settings ***
Library    %{ROBOTPYTHONSITEPACKAGESPATH}/PrometheusInterface/prometheus_interface.py    port_number=${8002}
\end{robotcode}

Every testsuite needs it's own robot or resoure file in which this import happens!

To support a better readibility of the test code we recommend to import the interface library with a certain name:

\begin{robotcode}
*** Settings ***
Library    %{ROBOTPYTHONSITEPACKAGESPATH}/PrometheusInterface/prometheus_interface.py    port_number=${8001}    WITH NAME    rf.prometheus_interface
\end{robotcode}

With \rcode{rf} is the abbreviation of \textbf{Robot Framework}.

% --------------------------------------------------------------------------------------------------------------

\newpage

\section{Support of Prometheus metric types}

\subsection{Counters and gauges}

The difference between a counter and a gauge is: A counter can be incremented only, whereas a gauge can be incremented, decremented and set to a certain value explicitly.
Both counters and gauges have to be added before.

A counter is added in this way:

\begin{robotcode}
rf.prometheus_interface.add_counter    name=(name of counter)     description=(description of counter)     labels=(label names)
\end{robotcode}

\rcode{name} and \rcode{decription} are required, \rcode{labels} is optional.

\vspace{2ex}

\textbf{Example:}

We want to count \textit{passed}, \textit{failed} and \textit{unknown} tests. A posssible definition of counter can look like this:

\begin{robotcode}
rf.prometheus_interface.add_counter    name=num_passed     description=number of passed tests     labels=room;testbench;testname
rf.prometheus_interface.add_counter    name=num_failed     description=number of failed tests     labels=room;testbench;testname
rf.prometheus_interface.add_counter    name=num_unknown    description=number of unknown tests    labels=room;testbench;testname
\end{robotcode}

In this example we assume that several different tests are executed in several rooms and on several testbenches. It is not necessary to add
a separate counter for every test on every testbench in every room. Only one counter (counting \textit{passed} tests) is required,
but with several labels. Every label will be a separate series of measurements of the corresponding counter. Every label is also a filter criteria
in \textbf{Grafana} when configuring metrics.

During the lifetime of a testsuite a counter can be added only once. Therefore we suggest to place the adding into an \rlog{__init__.robot} file.

\vspace{2ex}

A counter can be incremented (by default value 1) in this way:

\begin{robotcode}
rf.prometheus_interface.inc_counter    name=(name)     labels=(label values)
\end{robotcode}

\vspace{2ex}

A counter can also be incremented by a user defined value:

\begin{robotcode}
rf.prometheus_interface.inc_counter    name=(name)     value=(user defined increment)     labels=(label values)
\end{robotcode}

In \rcode{add_counter}, \rcode{labels} is a semicolon separated list of label \textit{names}. In \rcode{inc_counter}, \rcode{labels} is
a semicolon separated list of label \textit{values}. Label names and label values must fit together in \rcode{add_counter} and \rcode{inc_counter}.

\vspace{2ex}

\textbf{Example:}

\begin{robotcode}
rf.prometheus_interface.inc_counter    name=num_passed    labels=Room_1;Testbench 2;Suite-A-Test-01
\end{robotcode}

\vspace{2ex}

The same with gauges. A gauge is added in this way:

\begin{robotcode}
rf.prometheus_interface.add_gauge    name=(name of gauge)     description=(description of gauge)     labels=(label names)
\end{robotcode}

\rcode{name} and \rcode{decription} are required, \rcode{labels} is optional.

\vspace{2ex}

A gauge can e.g. be set to a certain value

\begin{robotcode}
rf.prometheus_interface.set_gauge    name=(name of gauge)    value=(value of gauge)    labels=(label values)
\end{robotcode}

As with counters, \rcode{labels} is a semicolon separated list of names or values.










